%Analysis_methods.tex
\documentclass [12pt, a4paper, twoside]  {article}
\usepackage [a4paper, left=30mm, right=30mm, top=20mm, bottom=20mm] {geometry}

%%% Paragraph formatting
\setlength{\parindent}{0pt} %new paragraph indent is 0
\setlength{\parskip}{1em}	%new paragraphs are indicated by a new line
\renewcommand{\baselinestretch}{1.2}

%%% Making sure line numbers are included
\RequirePackage{lineno}
\usepackage{lineno}

%%% Arranging graphics parameters
\usepackage [utf8] {inputenc}
% use this to put captions to the right of figures
\usepackage{sidecap}
\sidecaptionvpos{table}{c}
  
% allows exact specification of where figures are placed in the main text
\RequirePackage{float}
\usepackage{float}

\RequirePackage{ graphicx}
\usepackage { graphicx }
\graphicspath{ {./figs/} } %## Commented out because the path doesn't work (7/10/17)
\usepackage[labelfont=bf]{caption}
\DeclareCaptionLabelFormat{bf-parens}{\textbf{#1#2}}\captionsetup{labelsep=quad}
\usepackage{subcaption}
\usepackage{gensymb}
\usepackage{enumitem}
\setlist{nosep}

%%% Bibliography formatting
%% Remember that, in order to get in-text citations and the correct bibliography in the final PDF, you need to do a LaTeX (or XeLaTeX) run, then a BibTex run, followed by two more (Xe)LaTex runs. This should then solve any issues with in-text citations appearing as (?)
\usepackage{filecontents}
\usepackage[round]{natbib}	% See https://gking.harvard.edu/files/	natnotes2.pdf for more information

\usepackage{array}

\usepackage{multicol}
\usepackage{multirow}

%%% Ensures that the text will be crossreffed with hyperlinks
\usepackage{hyperref}
	\hypersetup{
		colorlinks=true
		linkcolor=blue
		filecolor=magenta
		urlcolor=true}

%% My own commands to make things a bit quicker to type
\newcommand{\super}{\textsuperscript}
\newcommand{\sub}{\textsubscript}

\begin{document}

Statistical analyses were carried out in R v3.5.0 \citep{R}, and graphics were generated using \texttt{r-base} and the \texttt{ggplot2} package \citep{ggplot2}. Data and analyses are available at this paper's Github page.

\section*{Sequence analysis}
Twelve clones from each replicate at the three timepoints considered had their two CRISPR loci analysed using PCR to determine spacer acquisition. Clones identified as carrying spacers then had the locus of interest sent sent for Sanger sequencing (Source Bioscience, UK). CRISPR1 and CRISPR3 were sequenced using primers [CR1 primer sequence] and [CR3 primer sequence]. The entire sequenced locus was mapped against the $\phi 2972$ genome (Accession: NC\textunderscore 007019.1) using BLAST. Putative spacer sequences were hits approximately 30bp. Genomic location and read direction of each protospacer hit were parsed using Geneious v9.1.8 \citep{kearse2012geneious}. False positive differences in hit location between putative spacer sequences generated by the BLAST algorithm were identified and the data cleaned accordingly. 

\section*{Phage survival}
Phage survival over the course of the experiment was analysed using a Cox proportional hazards model from the \texttt{survival} package \citep{survival}. Hazard ratio coefficients express the relative risk of phage extinction over time. 

\section*{Evolution of infectivity and resistance}
To measure the evolution of phage infectivity in terms of host range, we first calculated the proportion of host genotypes infected by a given phage genotype for each replicate at each timepoint. To measure the evolution of host resistance, the proportion of phage genotypes resisted by a given host genotype was also calculated. Infectivity or resistance were analysed in a Generalized Linear Model (GLM) with genotype as a fixed effect and a binomial family with a logit link function. Mean infectivity or resistance was then analysed for each timepoint in a Generalized Linear Mixed Model (GLMM) using the \texttt{lme4} package \citep{lme4}, with timepoint as a fixed effect and replicate as random effect. Model coefficients and confidence intervals were transformed from logits to probabilities prior to presentation. The final model was selected from candidates based on the reduction of heteroskedacity, $\chi ^2$ tests, and log-likelihood and AIC comparisons \citep{akaike1973,burnham2003model,burnham2004aic}.

\section*{Coevolutionary dynamics}
To infer if and what kind of coevolution occurred in our experiment, we conducted a time-shift assay by challenging hosts from each timepoint against phage from each timepoint, within each replicate. Phage were therefore tested against hosts from the phages' past, present, or future (Table 1).

% Table 1
\begin{table}[!htb]
\centering
\begin{tabular}{c|c|c|c|c}
 	 & \multicolumn{4}{c}{Phage} \\\cline{2-5}
	\multirow{4}{*}{Host} & & 1 & 4 & 9 \\\hline
	& 1 & Present & Past & Past \\\cline{2-5}
	 & 4 & Future & Present & Past \\\cline{2-5}
	& 9 & Future & Future & Present \\

\end{tabular}
\caption{Pairwise challenges between phage and hosts in the time-shift assay. Past, present or future refer to if hosts were contemporaneous or not with respect to the phage}
\end{table}

We tested if infectivity or resistance depended on host background i.e.\ if they were from phages' past, present, or future, first across the entire dataset and then within each timepoint. Infectivity was measured as the proportion of successful infections, with host resistance necessarily being $1-$ this value. Infectivity or resistance was analysed in a GLMM with host background (Environment; E) as a fixed effect and the interaction between host background and phage genotype (Genotype X Environment; GxE) as a random effect. Models had a binomial family with a logit link function, with coefficients and confidence intervals transformed to probabilities prior to presentation.

To test for the relative importance of arms race versus fluctuating selection in our experiment, we estimated the strength of the GxE effect on infectivity and resistance following \citep{hall2011coevophage}. Under a simple arms race, phage should be more infective to hosts from their past compared to their present or future, independent of phage genotype. By contrast, under fluctuating selection phage genotypes will differ in their infectivity to hosts from their past, present or future. Differences among phage genotypes are therefore detectable as the proportion of the host environment (E) residual variance explained by the interaction between host environment and phage genotype (GxE). Increasing values of this proportion relate to increasing differences among phage genotypes. We estimated this by calculating the ratio of the mean square (MS) of an E-only model to the MS of a GxE model for each replicate at each timepoint. Ratios were then analysed in a GLMM with timepoint as a fixed effect and replicate as a random effect, with a normal family and identity link function. Residuals were square-root transformed to fit the assumption of normality. 

\clearpage
%% Set up the bibliography	
\bibliographystyle{./humannatemph.bst}				% default "author-year" style for natbib
\bibliography{../../../references.bib}

\end{document}
